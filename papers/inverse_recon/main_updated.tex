\\documentclass[11pt]{article}

% ==================== Packages ====================
\\usepackage[T1]{fontenc}
\\usepackage{lmodern}
\\usepackage[margin=1in]{geometry}

\\usepackage{amsmath,amssymb,amsthm}
\\usepackage{mathtools}
\\usepackage{booktabs}
\\usepackage{enumitem}
\\usepackage{microtype}
\\usepackage[hidelinks]{hyperref} % keep near the end

% --- table helpers ---
\\usepackage{array}
\\usepackage{tabularx}
\\usepackage{makecell}
\\usepackage{threeparttable}

% --- graphics ---
\\usepackage{graphicx}
% Make compilation robust whether invoked from repo root or from update_paper/.
\\graphicspath{{./}{update_paper/}{../update_paper/}{FINAL_FIGURES/}{FINAL_FIGURES/generated/}{LAST_FIGURE/}{../FINAL_FIGURES/}{../FINAL_FIGURES/generated/}{../LAST_FIGURE/}}

% ==================== Environments ====================
\\newtheorem{definition}{Definition}
\\newtheorem{remark}{Remark}
\\newtheorem{proposition}{Proposition}
\\newtheorem{conjecture}{Conjecture}

% ==================== Macros ====================
\\newcommand{\\dd}{\\mathrm{d}}
\\newcommand{\\cF}{\\mathcal{F}}
\\newcommand{\\bbR}{\\mathbb{R}}
\\newcommand{\\bbN}{\\mathbb{N}}

\\title{Inverse Reconstruction of an Effective Horizon--Entropy Slope Deformation from Late-Time Data\\\\
\\large A Calibrated Nonparametric Pipeline with Mapping and Perturbation-Sector Sensitivity Control}
\\author{Aiden B. Smith}
\\date{January 30, 2026}

\\begin{document}
\\maketitle

\\begin{abstract}
We present a calibrated, nonparametric inverse-inference pipeline that reconstructs an \\emph{effective}
horizon-entropy \\emph{slope} deformation function from late-time cosmological data, rather than fitting
a pre-chosen parametric entropy model. The primary inferred object is an entropy-slope deformation function $\\mu(A)$
(defined in Eq.~\\eqref{eq:mu_def}), reconstructed from SN/BAO/CC data and, in extended ``info+''
configurations, from growth information and CMB lensing constraints under explicitly enumerated
mapping variants that encode ``first law'' ambiguities.

We emphasize validation and reproducibility: simulation-based calibration (SBC) under a
Bekenstein--Hawking truth $\\mu=1$, explicit Jacobians for hard-bounded parameters, synthetic-data
generator audits, and seed robustness. The reconstructed function is summarized by scalar
``scar'' statistics: a weighted mean deviation $m$ and a weighted slope statistic $s$, alongside
posterior bands for $\\mu(A)$.

As a clean reproducibility check of the proxy-stack baseline (FSBAO + compressed Planck lensing proxy;
mapping M0), we re-ran the earlier exploratory benchmark on a clean commit and obtain
$m=-0.020\\pm 0.183$ with $P(m>0)=0.415$, and $s=-0.480\\pm 0.356$ with $P(s>0)=0.083$.
Across three additional clean M0 seeds, the slope mean remains negative ($s\\in[-0.623,-0.480]$) with
$P(s>0)\\in[0.048,0.092]$, while $m$ remains consistent with zero.
Extending the same proxy stack to mapping-variant sensitivity tests with four seeds each, we find
that adding residual-closure freedom (M1) reduces the magnitude of the negative slope (seed means
$s\\in[-0.400,-0.182]$), whereas the curved-horizon mapping (M2) yields slopes similar to M0 (seed
means $s\\in[-0.531,-0.448]$), albeit with a robust overlap-domain procedure required for scalar
summaries in all current M2 runs.
An extended-$z$ single-seed check to $z_{\\max}=0.77$ weakens the slope to $s=-0.325\\pm 0.275$.

This mapping sensitivity indicates that the negative-slope scar is partially degenerate with closure ambiguity and is therefore treated as a systematic sensitivity signal, not as unique evidence for an entropy deformation.

In a pilot full-likelihood ``info+'' multistart suite (mapping M0; SN+CC+BAO distance-only + DR16 LRG $f\\sigma_8$ points + Planck 2018 $C_\\ell^{\\phi\\phi}$ bandpowers + Shapefit full-shape $P(k)$), five independent seeds yield mixed-sign $m$ values with an across-seed mean $-0.039$ (across-seed s.e. $0.090$), while the slope mean is negative in all seeds with an across-seed mean $-0.334$ (across-seed s.e. $0.059$) and per-seed $P(s>0)\\in[0.115,0.339]$. Relative to the proxy-stack configuration (typical $|s|\\sim 0.5$), adding RSD, full Planck lensing bandpowers, and full-shape $P(k)$ anchors ``squeezes'' the tilt toward $s=0$ without eliminating the negative mean, suggesting a residual, configuration-dependent late-time preference rather than an unconstrained flexible mode. Integrated-autocorrelation warnings (chain shorter than $50\\,\\tau$) indicate that these full-likelihood runs remain pilot-quality until validated with longer chains and SBC.
All slope statements are explicitly configuration-dependent and are interpreted as phenomenological
consistency information until validated with evidence estimates, full-information multistarts, and
SBC.
\\end{abstract}

\\tableofcontents

% ============================================================
\\section{Introduction}
We ask whether late-time cosmological data can support an \\emph{effective} deformation of the
horizon-entropy \\emph{slope}, without assuming a parametric Tsallis/Barrow/Kaniadakis form.
The core idea is to invert the map from $\\mu(A)$ to observables (e.g.\\ $H(z)$ and distances) and
to infer $\\mu(A)$ directly from data.

We treat this reconstruction as an \\emph{effective} inference problem. In particular, when we add
growth and CMB lensing information, we explicitly separate:
(i) background-driven effects (how $\\mu(A)$ modifies $H(z)$), from
(ii) perturbation-sector assumptions (how growth and lensing respond).
This framing turns the ``info+'' configuration into a controlled \\emph{consistency test} rather than
a claim of full microphysical reconstruction.

This work contributes:
\\begin{enumerate}[leftmargin=2.0em]
\\item A nonparametric inverse reconstruction of $\\mu(A)$, with post hoc proximity tests to parametric
families only after inference.
\\item A calibrated inference pipeline: SBC under $\\mu=1$, explicit hard-bound transforms with
Jacobian factors, and synthetic-data generator audits.
\\item Explicit mapping-variant control (M0/M1/M2), enabling sensitivity tests to ``first law''
ambiguities in the thermodynamic map.
\\item Robustness checks via multi-starts, seeds, and likelihood configuration changes.
\\end{enumerate}

For reader convenience, we summarize the paper structure. Section~2 introduces the thermodynamic forward map that connects an effective horizon--entropy slope deformation $\\mu(A)$ to the background expansion history through an ODE for $u(z)\\equiv H(z)^2$ and standard distance relations. Section~3 enumerates the mapping variants (M0/M1/M2) that encode controlled ``first-law'' ambiguities, including residual-closure freedom and a curvature-dependent horizon-area map, and records a minimal perturbation-sector embedding used only for out-of-sample targets. Section~4 describes the inference methodology (PTEMCEE sampling, hard-bound transforms, and prior construction) together with reproducibility bookkeeping. Section~5 presents simulation-based calibration results for the forward inference pipeline under a $\\mu=1$ truth. Section~6 presents the main results, organized into a clean proxy-stack baseline (with mapping/seed robustness) versus an extended ``info+'' configuration that adds growth and full lensing/full-shape information as an explicitly labeled consistency test.

% ============================================================
\\section{Forward model and mapping}
\\subsection{Definition of $\\mu(A)$}
We define the entropy-slope deformation
\\begin{equation}
\\mu(A)\\coloneqq
\\frac{(\\dd S/\\dd A)_{\\mathrm{BH}}}{\\dd S/\\dd A}.
\\label{eq:mu_def}
\\end{equation}


\\subsection{Log-area coordinate and log-deformation}
We work in a log-area coordinate centered at the present-epoch horizon area:
\\begin{equation}
x \\coloneqq \\log\\!\\left(\\frac{A}{A_0}\\right),
\\qquad
g(x) \\coloneqq \\log \\mu(A)\\quad (A=A_0 e^{x}).
\\label{eq:gx}
\\end{equation}


\\subsection{Forward map from $\\mu(A)$ to $H(z)$}
We define
\\begin{equation}
u(z)\\coloneqq H(z)^2,
\\label{eq:u_def}
\\end{equation}
and use a Cai--Kim/Clausius-style late-time mapping (matter-dominance approximation for $(\\rho+p)$)
to obtain the ODE
\\begin{equation}
\\frac{\\dd u}{\\dd z}
=
3 H_0^2 \\Omega_{m0} (1+z)^2\\,\\mu\\!\\bigl(A(z)\\bigr).
\\label{eq:forward_u}
\\end{equation}
An explicit Clausius derivation consistent with the sign and normalization used in the implementation is given in Appendix~\\ref{app:clausius_derivation}.
The apparent-horizon area mapping is
\\begin{equation}
A(z)
=
\\frac{4\\pi c^2}{H(z)^2 - \\Omega_{k0} H_0^2 (1+z)^2},
\\qquad
H(z)^2 - \\Omega_{k0} H_0^2 (1+z)^2 > 0.
\\label{eq:area_general}
\\end{equation}
For $\\Omega_{k0}=0$, this reduces to $A(z)=4\\pi c^2/H(z)^2$.

We also define the apparent-horizon radius $R_A(z)$ by $A(z)=4\\pi R_A(z)^2$:
\\begin{equation}
R_A(z)\\coloneqq \\sqrt{\\frac{A(z)}{4\\pi}}
=
\\frac{c}{\\sqrt{H(z)^2 - \\Omega_{k0} H_0^2 (1+z)^2}}.
\\label{eq:RA_def}
\\end{equation}


\\subsection{Distances}
Distances are computed from $H(z)$ using standard FRW relations:
\\begin{align}
D_C(z) &= c \\int_0^z \\frac{\\dd z'}{H(z')}, \\label{eq:DC}\\\\
D_M(z) &=
\\begin{cases}
\\frac{c}{H_0\\sqrt{\\Omega_{k0}}}\\,
\\sinh\\!\\left(\\sqrt{\\Omega_{k0}}\\; \\frac{H_0 D_C(z)}{c}\\right), & \\Omega_{k0}>0,\\\\[6pt]
D_C(z), & \\Omega_{k0}=0,\\\\[6pt]
\\frac{c}{H_0\\sqrt{|\\Omega_{k0}|}}\\,
\\sin\\!\\left(\\sqrt{|\\Omega_{k0}|}\\; \\frac{H_0 D_C(z)}{c}\\right), & \\Omega_{k0}<0,
\\end{cases}
\\label{eq:DM}
\\end{align}
with angular diameter distance $D_A(z)=D_M(z)/(1+z)$ and luminosity distance
$D_L(z)=(1+z)\\,D_M(z)$.


% ============================================================
\\section{Mapping variants (M0/M1/M2)}
We implement three mapping variants to capture ``first law'' ambiguities and allow sensitivity
tests.

\\paragraph{M0 (baseline).}
Baseline mapping with $\\Omega_{k0}=0$ and no residual closure term:
\\begin{equation}
\\frac{\\dd u}{\\dd z}
=
3 H_0^2 \\Omega_{m0} (1+z)^2\\,\\mu\\!\\bigl(A(z)\\bigr),
\\qquad
A(z) = \\frac{4\\pi c^2}{u(z)}.
\\label{eq:M0}
\\end{equation}


\\paragraph{M1 (residual closure).}
Adds a smooth residual $R(z)$ to capture closure ambiguities:
\\begin{equation}
\\frac{\\dd u}{\\dd z}
=
3 H_0^2 \\Omega_{m0} (1+z)^2\\,\\mu\\!\\bigl(A(z)\\bigr)\\,[1+R(z)].
\\label{eq:M1}
\\end{equation}
We parameterize $R(z)$ with a low-order spline (e.g.\\ $N_R$ knots on $[0,z_{\\max}]$), with Gaussian
priors at the knots and a smoothness prior on second differences.

\\paragraph{M2 (curved-horizon mapping).}
Promotes $\\Omega_{k0}$ to a nuisance parameter and updates $A(z)$ accordingly:
\\begin{equation}
\\frac{\\dd u}{\\dd z}
=
3 H_0^2 \\Omega_{m0} (1+z)^2\\,\\mu\\!\\bigl(A(z)\\bigr),
\\qquad
A(z) = \\frac{4\\pi c^2}{u(z) - \\Omega_{k0} H_0^2 (1+z)^2}.
\\label{eq:M2}
\\end{equation}

\\begin{figure}[h!]
\\centering
\\includegraphics[width=0.88\\linewidth]{horizon_radius_mapping_demo.pdf}
\\caption{Illustration of the curvature-dependent apparent-horizon map $A(z)$ (Eq.~\\eqref{eq:area_general}) through the horizon radius $R_A(z)$ for representative $\\Omega_{k0}$ values.}
\\label{fig:horizon_radius_mapping_demo}
\\end{figure}

\\begin{remark}[Nomenclature note]
Some internal batch scripts reserve the label ``M3'' for additional experimental mappings (e.g.\\ alternative horizon definitions or perturbation-sector couplings). Results reported here use only M0--M2.
\\end{remark}


\\subsection{Minimal perturbation embedding for out-of-sample targets}
\\label{sec:minimal_embedding}
Our reconstruction maps $\\mu(A)$ into background observables through Eqs.~\\eqref{eq:forward_u}--\\eqref{eq:DM}. To define falsifiable out-of-sample targets that probe the perturbation/propagation sector, we record a minimal embedding in which the horizon entropy slope is set by an effective gravitational coupling,
\\begin{equation}
\\frac{\\dd S}{\\dd A} = \\frac{1}{4\\,G_{\\mathrm{eff}}(A)},
\\qquad \\Rightarrow\\qquad
\\mu(A) = \\frac{G_{\\mathrm{eff}}(A)}{G_N},
\\label{eq:Geff_mu}
\\end{equation}
so that $G_{\\mathrm{eff}}(z)=G_N\\,\\mu(A(z))$. Interpreting this as a running Planck mass $M_*^2\\propto 1/G_{\\mathrm{eff}}$, one has $M_*^2(z)\\propto 1/\\mu(A(z))$ and therefore an EFT-of-DE Planck-mass running
\\begin{equation}
\\alpha_M(z)\\equiv \\frac{\\dd\\ln M_*^2}{\\dd\\ln a}
=-\\frac{\\dd\\ln\\mu}{\\dd\\ln a}.
\\label{eq:alphaM}
\\end{equation}

\\paragraph{Local time-variation proxy.}
At $z=0$, the implied fractional time variation is
\\begin{equation}
\\frac{\\dot G_{\\mathrm{eff}}}{G_{\\mathrm{eff}}}
=
\\frac{\\dd \\ln G_{\\mathrm{eff}}}{\\dd t}
=
\\frac{\\dd\\ln\\mu}{\\dd t}
=
-(H\\,\\alpha_M)\\big|_{z=0},
\\label{eq:dotG_over_G}
\\end{equation}
which provides a concrete local-consistency stress test under this minimal identification.
This is not a direct cosmological standard-siren observable: it tests only whether the
minimal unscreened identification of local and cosmological couplings is internally consistent.

\\begin{figure}[h!]
\\centering
\\includegraphics[width=0.92\\linewidth]{proxy_M0_seed123_dotG_over_G_hist.pdf}
\\caption{Consistency check under the minimal $\\alpha_M$-only embedding: posterior for the present-day time variation $\\dot G_{\\mathrm{eff}}/G_{\\mathrm{eff}}$ implied by the reconstructed $\\mu(A)$, shown here for the proxy-stack clean M0 seed 123 reconstruction.}
\\label{fig:dotG_over_G_constraints}
\\end{figure}

\\begin{remark}[Local-constraint caveat for the minimal embedding]
Constraints on present-day time variation of Newton's constant are very strong. If the reconstructed
$\\mu(A)$ posterior implies a present-day drift $\\dot G_{\\mathrm{eff}}/G_{\\mathrm{eff}}$ that violates
local bounds when interpreted through Eq.~\\eqref{eq:dotG_over_G}, then the $\\alpha_M$-only embedding
is ruled out (even if the background reconstruction remains a valid phenomenological description).
In that case, any physical completion would require either screening/decoupling between local and
cosmological couplings or additional perturbation-sector structure beyond the minimal ansatz.
\\end{remark}

\\paragraph{Standard-siren propagation target.}
In running-Planck-mass phenomenology with luminal GW speed, the standard-siren distance ratio obeys the well-known relation
\\begin{equation}
\\frac{d_L^{\\mathrm{GW}}(z)}{d_L^{\\mathrm{EM}}(z)}
=
\\frac{M_*(0)}{M_*(z)}
=
\\sqrt{\\frac{\\mu(A(z))}{\\mu(A(0))}}.
\\label{eq:dlGW_over_dlEM}
\\end{equation}

\\begin{figure}[h!]
\\centering
\\includegraphics[width=0.92\\linewidth]{proxy_M0_seed123_dLgw_over_dLem_band.pdf}
\\caption{Proxy-stack clean M0 seed 123: standard-siren propagation target under the minimal $\\alpha_M$-only embedding, shown as the posterior band of $d_L^{\\mathrm{GW}}/d_L^{\\mathrm{EM}}$ implied by Eq.~\\eqref{eq:dlGW_over_dlEM}.}
\\label{fig:dlGW_over_dlEM_band}
\\end{figure}

\\paragraph{Out-of-sample prediction: a void-lensing amplitude proxy (Tier 1).}
As an additional lightweight falsification probe under the same $\\alpha_M$-only embedding, we map
posterior draws of $\\mu(A)$ into a crude, amplitude-only proxy intended to capture the leading
dependence of void-lensing-type signals on an effective Newton coupling.
In this Tier~1 construction, the proxy amplitude is a redshift-weighted average of the coupling
ratio $G_{\\mathrm{eff}}(z)/G_{\\mathrm{eff}}(0)=\\mu(A(z))/\\mu(A(0))$; values above unity correspond to an
effective strengthening relative to today over the chosen window.

For a chosen void-sample redshift window $z\\in[z_{\\min},z_{\\max}]$, define for each posterior draw
$(j)$
\\begin{equation}
\\mathcal{A}_{\\mathrm{void}}^{(j)} \\coloneqq
\\int_{z_{\\min}}^{z_{\\max}} w_V(z)\\,
\\frac{G_{\\mathrm{eff}}^{(j)}(z)}{G_{\\mathrm{eff}}^{(j)}(0)}\\,\\dd z
=
\\int_{z_{\\min}}^{z_{\\max}} w_V(z)\\,
\\frac{\\mu^{(j)}(A(z))}{\\mu^{(j)}(A(0))}\\,\\dd z,
\\qquad
\\int_{z_{\\min}}^{z_{\\max}} w_V(z)\\,\\dd z = 1,
\\label{eq:void_amp_proxy}
\\end{equation}
where $w_V(z)\\ge 0$ is a user-chosen weight.
Using Eq.~\\eqref{eq:dlGW_over_dlEM}, the integrand can equivalently be written as the square of the
standard-siren distance ratio:
$\\mu(A(z))/\\mu(A(0))=[d_L^{\\mathrm{GW}}(z)/d_L^{\\mathrm{EM}}(z)]^2$.

\\begin{figure}[h!]
\\centering
\\includegraphics[width=0.92\\linewidth]{void_amp_proxy_schematic.pdf}
\\caption{Conceptual illustration for the Tier~1 void-lensing proxy. The full analysis requires a profile-based template and a verified measurement definition; this schematic is included to make the sign logic and the role of $G_{\\mathrm{eff}}$ intuitive.}
\\label{fig:void_schematic}
\\end{figure}

\\paragraph{Minimal growth modification under this embedding (not used in Results).}
In GR, the linear growth factor $D(a)$ (normalized at $a=1$) obeys
\\begin{equation}
\\frac{\\dd^2 D}{\\dd a^2}
+
\\left(
\\frac{3}{a}
+
\\frac{\\dd \\ln H}{\\dd a}
\\right)
\\frac{\\dd D}{\\dd a}
-
\\frac{3}{2}\\,
\\frac{\\Omega_{m0} H_0^2}{a^5\\,H(a)^2}\\,D
=0.
\\label{eq:growth_ode}
\\end{equation}
If one further assumes that the same $G_{\\mathrm{eff}}$ controls the strength of clustering in the
subhorizon Poisson equation (and assumes negligible gravitational slip, $\\Phi=\\Psi$), then the GR
growth equation is modified only by a rescaling of the source term:
\\begin{equation}
\\frac{\\dd^2 D}{\\dd a^2}
+
\\left(
\\frac{3}{a}
+
\\frac{\\dd \\ln H}{\\dd a}
\\right)
\\frac{\\dd D}{\\dd a}
-
\\frac{3}{2}\\,
\\frac{\\Omega_{m0} H_0^2}{a^5\\,H(a)^2}\\,
\\mu\\!\\bigl(A(a)\\bigr)\\,D
=0,
\\label{eq:growth_ode_embed}
\\end{equation}
where $A(a)$ denotes the mapped area evaluated at $z(a)=a^{-1}-1$.
This extension is deferred to future work; the present paper treats growth/lensing likelihoods as controlled anchors under explicitly stated assumptions rather than as a unique perturbation reconstruction.


% ============================================================
\\section{Inference algorithm}
\\subsection{Sampler}
We sample the posterior using PTEMCEE (parallel-tempered ensemble MCMC). In all production runs we use a fixed temperature ladder with $n_T$ temperatures and maximum temperature $T_{\\max}$, and we report posterior results from the cold chain only ($T=1$). After discarding burn-in, the cold chain is flattened over walkers and steps and a fixed number of draws is selected uniformly at random without replacement for downstream post-processing (posterior bands, scar summaries, and derived predictions). Exact $(n_T,T_{\\max},N_w,N_{\\mathrm{step}},N_{\\mathrm{burn}},N_{\\mathrm{draw}})$ values are documented per configuration in Section~\\ref{sec:results}.

\\subsection{Hard-bounded parameter transforms}
For a parameter $\\theta$ constrained to an interval $[\\theta_{\\min},\\theta_{\\max}]$, we sample an
unconstrained variable $u\\in\\bbR$ and transform via a logistic map
\\begin{equation}
\\theta(u)=\\theta_{\\min}+\\frac{\\theta_{\\max}-\\theta_{\\min}}{1+e^{-u}},
\\label{eq:logistic}
\\end{equation}
with Jacobian
\\begin{equation}
\\left|\\frac{\\dd \\theta}{\\dd u}\\right|
=
(\\theta_{\\max}-\\theta_{\\min})\\,
\\frac{e^{-u}}{(1+e^{-u})^2}.
\\label{eq:jacobian}
\\end{equation}
The log-Jacobian term $\\log|\\dd\\theta/\\dd u|$ is added to the log posterior.


\\subsection{Baseline prior bounds and nuisance priors (as implemented)}
For transparency, Table~\\ref{tab:priors_baseline} lists the hard bounds and the explicit nuisance/hyperparameter priors used in the clean proxy-stack baseline.

\\begin{table}[h!]
\\centering
\\small
\\setlength{\\tabcolsep}{6pt}
\\begin{tabular}{@{}lcc@{}}
\\toprule
Quantity & Prior / bound & Comment \\\\
\\midrule
$H_0$ & $[40,100]$ & hard bound (km/s/Mpc) \\\\
$\\Omega_{m0}$ & $[0.2,0.4]$ & hard bound \\\\
$r_d$ & $[120,170]$ & hard bound (Mpc) \\\\
$\\sigma_{8,0}$ & $[0.6,1.0]$ & hard bound \\\\
$\\sigma_{\\mathrm{cc,jit}}$ & $\\mathrm{HalfNormal}(\\text{scale}=10)$ & CC jitter (km/s/Mpc) \\\\
$\\sigma_{\\mathrm{sn,jit}}$ & $\\mathrm{HalfNormal}(\\text{scale}=0.05)$ & SN jitter (mag) \\\\
$\\log\\sigma_{d2}$ & $[-12,3]$ & hard bound on smoothness scale (log-space) \\\\
$\\sigma_{d2}$ & $\\mathrm{HalfNormal}(\\text{scale}=0.185)$ & truncated by $\\log\\sigma_{d2}$ bound \\\\
\\bottomrule
\\end{tabular}
\\caption{Baseline prior bounds and nuisance/hyperparameter priors used in the clean proxy-stack rerun. In the implementation the scale parameters are sampled in log space and include the appropriate Jacobian terms.}
\\label{tab:priors_baseline}
\\end{table}

\\subsection{Diagnostics}
We report acceptance fractions, integrated autocorrelation time (IAT), and effective sample sizes
(ESS) for key scalar projections.


% ============================================================
\\section{Simulation-based calibration and coverage}
We assess calibration by generating synthetic datasets under a known truth (here $\\mu=1$ with a fiducial cosmology) and verifying rank uniformity and nominal coverage.

\\paragraph{SBC configuration (BH truth).}
We use $N=256$ replicates under a Bekenstein--Hawking truth ($\\mu=1$), with $z_{\\max}=0.77$ and the geometry-only forward inference stack (binned Pantheon+ SN, cosmic chronometers, and BAO). Each replicate uses PTEMCEE with $n_T=4$, $T_{\\max}=25$, $N_w=32$, $N_{\\mathrm{step}}=5000$, and burn-in $N_{\\mathrm{burn}}=2000$, recording $N_{\\mathrm{draw}}=400$ posterior draws.

\\paragraph{Rank histograms and coverage.}
Figure~\\ref{fig:sbc_rank_hist} shows SBC rank histograms for $H_0$, $\\Omega_{m0}$, and the slope scar $s$. Table~\\ref{tab:sbc_coverage} summarizes empirical coverage for selected parameters at nominal 68\\% and 95\\% credible levels.

\\begin{figure}[h!]
\\centering
\\includegraphics[width=0.92\\linewidth]{figure_C_sbc_rank_hist.png}
\\caption{Simulation-based calibration (BH truth, $\\mu=1$): rank histograms for $H_0$, $\\Omega_{m0}$, and the slope scar statistic $s$. The gray band indicates the 99\\% expected range per bin under perfect uniformity (for the chosen binning).}
\\label{fig:sbc_rank_hist}
\\end{figure}

\\begin{table}[h!]
\\centering
\\small
\\setlength{\\tabcolsep}{8pt}
\\begin{tabular}{@{}lcc@{}}
\\toprule
Quantity & coverage (68\\%) & coverage (95\\%) \\\\
\\midrule
$m$ (scar mean) & 0.453 & 0.617 \\\\
$s$ (scar slope) & 0.410 & 0.711 \\\\
$H_0$ & 0.477 & 0.727 \\\\
$\\Omega_{m0}$ & 0.676 & 0.949 \\\\
$r_d$ & 0.496 & 0.793 \\\\
\\bottomrule
\\end{tabular}
\\caption{SBC (BH truth, geometry-only forward inference): empirical coverage for selected parameters. The scar summaries are substantially under-covered in this configuration.}
\\label{tab:sbc_coverage}
\\end{table}

\\paragraph{Implication for ``negative slope'' significance.}
In this SBC configuration, using a one-sided ``negative slope'' rule $P(s>0)<\\alpha$ with $\\alpha=0.05$ yields a BH-null false-positive rate $\\mathrm{FPR}\\simeq 0.238$, indicating that naive posterior sign-probabilities for $s$ can substantially overstate significance until the calibration issue is resolved. Accordingly, throughout we treat sign preferences as descriptive bookkeeping and interpret scar summaries conservatively as configuration-dependent consistency information.


% ============================================================
\\section{Results}
\\label{sec:results}

\\subsection{Proxy-stack clean replacement and mapping sensitivity suite}
\\label{sec:results_proxy}
This section reports a clean, paper-grade replacement of the earlier proxy-stack baseline run
(FSBAO + compressed Planck lensing proxy; mapping M0) and a robustness suite: multi-seed repeats for
M0 (four seeds total), multi-seed mapping-variant repeats for M1 and M2 (four seeds each), and an
extended-redshift single-seed check. All runs in this section use commit SHA
\\texttt{32b23931ca199aa06078d2d9d88e333959946a4d}; the base clean rerun is recorded with
\\texttt{dirty=False}.

\\paragraph{Shared likelihood stack (proxy configuration).}
All runs in this section use the dataset stack in Table~\\ref{tab:datasets_proxy} over the automatic SN-density-selected redshift domain $z\\in[0.02,0.62]$. Planck lensing information enters only through a compressed proxy constraint (no full $C_\\ell^{\\phi\\phi}$ bandpowers are used in these proxy runs).

\\begin{table}[h!]
\\centering
\\footnotesize
\\setlength{\\tabcolsep}{4pt}
\\begin{tabularx}{\\textwidth}{@{}l c c X@{}}
\\toprule
Dataset component & points & redshift/scale & Source / notes \\\\
\\midrule
Pantheon+ (cosmology subset; stat+sys; $z_{\\mathrm{HD}}$) & 1322 (binned to 12) & $z\\in[0.02,0.62]$ & PantheonPlusSH0ES/DataRelease (commit \\texttt{c447f0f...}); full covariance with $M$-marginalization; binned for forward inference \\\\
Cosmic chronometers (BC03\\_all) & 9 & $z\\in[0.09,0.593]$ & baudren/montepython\\_public (tag 2.2); diagonal errors plus jitter $\\sigma_{\\mathrm{cc,jit}}$ \\\\
FSBAO (SDSS DR12 consensus FS) & 9 & $z=\\{0.38,0.51,0.61\\}$ & CobayaSampler/bao\\_data (commit \\texttt{bb0c1c9...}); $(D_M/r_s,\\,H r_s,\\,f\\sigma_8)$ with full covariance \\\\
FSBAO (SDSS DR16 LRG BAO+FSBAO) & 6 & $z=\\{0.38,0.51\\}$ & CobayaSampler/bao\\_data (commit \\texttt{bb0c1c9...}); $(D_M/r_s,\\,D_H/r_s,\\,f\\sigma_8)$ with full covariance (in-range subset) \\\\
BAO-only (DESI 2024 ``ALL'') & 3 & $z=\\{0.295,0.51\\}$ & CobayaSampler/bao\\_data (commit \\texttt{bb0c1c9...}); includes $D_V/r_s$ at $z=0.295$ and $(D_M/r_s,D_H/r_s)$ at $z=0.51$ \\\\
Planck CMB lensing proxy & 1 & -- & Gaussian proxy on $\\sigma_8\\,\\Omega_{m0}^{0.25}=0.589\\pm0.020$ parsed from Planck 2018 parameters paper source (arXiv:1807.06209) \\\\
\\bottomrule
\\end{tabularx}
\\caption{Datasets used in the proxy-stack configuration (geometry + FSBAO + compressed Planck lensing proxy). Counts reflect the automatically selected domain $z\\in[0.02,0.62]$.}
\\label{tab:datasets_proxy}
\\end{table}

\\paragraph{Shared sampler settings.}
Sampler: PTEMCEE with $n_T=8$ temperatures, $T_{\\max}=50$, and $N_w=64$ walkers. Steps/burn/draws
$=600/200/400$.

\\paragraph{Derived parameter $S_8$.}
For compact reporting, we use
\\begin{equation}
S_8\\coloneqq \\sigma_{8,0}\\left(\\frac{\\Omega_{m0}}{0.3}\\right)^{1/2}.
\\label{eq:S8_def}
\\end{equation}

\\subsubsection{Cosmological summary}
Table~\\ref{tab:proxy_cosmo_multirun} summarizes p50 medians for selected cosmological parameters
across mapping variants and seeds in the proxy configuration.

\\begin{table}[h!]
\\centering
\\footnotesize
\\setlength{\\tabcolsep}{4pt}
\\begin{tabularx}{\\textwidth}{@{}c r r *{3}{>{\\centering\\arraybackslash}X}@{}}
\\toprule
Map & seed & $z_{\\max}$ & $H_0$ p50 & $\\Omega_{m0}$ p50 & $S_8$ p50 \\\\
\\midrule
M0 & 101 & 0.62 & 70.365 & 0.337 & 0.817 \\\\
M0 & 123 & 0.62 & 70.615 & 0.317 & 0.808 \\\\
M0 & 202 & 0.62 & 70.732 & 0.343 & 0.825 \\\\
M0 & 303 & 0.62 & 70.739 & 0.325 & 0.814 \\\\
\\addlinespace
M1 & 101 & 0.62 & 70.400 & 0.285 & 0.773 \\\\
M1 & 123 & 0.62 & 71.341 & 0.306 & 0.799 \\\\
M1 & 202 & 0.62 & 69.012 & 0.306 & 0.799 \\\\
M1 & 303 & 0.62 & 70.682 & 0.291 & 0.785 \\\\
\\addlinespace
M2 & 101 & 0.62 & 70.660 & 0.316 & 0.800 \\\\
M2 & 123 & 0.62 & 70.126 & 0.330 & 0.814 \\\\
M2 & 202 & 0.62 & 69.701 & 0.335 & 0.817 \\\\
M2 & 303 & 0.62 & 70.306 & 0.342 & 0.810 \\\\
\\addlinespace
M0 & 123 & 0.77 & 68.993 & 0.332 & 0.816 \\\\
\\bottomrule
\\end{tabularx}
\\caption{Proxy-stack cosmological medians by mapping variant and seed. For compactness we report p50
medians; full marginal intervals are available in the machine-readable summaries for each run. The
final row is an extended-redshift single-seed check with $z_{\\max}=0.77$ (same proxy stack, enlarged
domain).}
\\label{tab:proxy_cosmo_multirun}
\\end{table}


\\subsubsection{Scar statistics}
Table~\\ref{tab:proxy_scars_multirun} summarizes scar statistics and overlap-domain bookkeeping in the
proxy configuration (sampler diagnostics are reported in the machine-readable summaries).

\\begin{table}[h!]
\\centering
\\footnotesize
\\setlength{\\tabcolsep}{4pt}
\\begin{tabularx}{\\textwidth}{@{}c r r c c *{3}{>{\\centering\\arraybackslash}X}@{}}
\\toprule
Map & seed & $z_{\\max}$ & log$A$ method & fallback &
$m$ (mean$\\pm$std) & $s$ (mean$\\pm$std) & $P(s>0)$ \\\\
\\midrule
M0 & 101 & 0.62 & strict & False & $-0.049\\pm0.160$ & $-0.486\\pm0.345$ & 0.065 \\\\
M0 & 123 & 0.62 & strict & False & $-0.020\\pm0.183$ & $-0.480\\pm0.356$ & 0.083 \\\\
M0 & 202 & 0.62 & strict & False & $-0.079\\pm0.152$ & $-0.623\\pm0.372$ & 0.048 \\\\
M0 & 303 & 0.62 & strict & False & $-0.047\\pm0.180$ & $-0.500\\pm0.372$ & 0.092 \\\\
\\addlinespace
M1 & 101 & 0.62 & strict & False & $-0.004\\pm0.171$ & $-0.242\\pm0.268$ & 0.158 \\\\
M1 & 123 & 0.62 & strict & False & $-0.064\\pm0.161$ & $-0.182\\pm0.271$ & 0.195 \\\\
M1 & 202 & 0.62 & strict & False & $-0.065\\pm0.132$ & $-0.400\\pm0.259$ & 0.037 \\\\
M1 & 303 & 0.62 & strict & False & $+0.004\\pm0.148$ & $-0.321\\pm0.318$ & 0.128 \\\\
\\addlinespace
M2 & 101 & 0.62 & robust & True & $-0.033\\pm0.199$ & $-0.487\\pm0.354$ & 0.070 \\\\
M2 & 123 & 0.62 & robust & True & $-0.054\\pm0.191$ & $-0.531\\pm0.324$ & 0.055 \\\\
M2 & 202 & 0.62 & robust & True & $-0.077\\pm0.194$ & $-0.501\\pm0.324$ & 0.058 \\\\
M2 & 303 & 0.62 & robust & True & $-0.098\\pm0.180$ & $-0.448\\pm0.374$ & 0.117 \\\\
\\addlinespace
M0 & 123 & 0.77 & strict & False & $-0.035\\pm0.142$ & $-0.325\\pm0.275$ & 0.110 \\\\
\\bottomrule
\\end{tabularx}
\\caption{Proxy-stack scar statistics by mapping variant and seed. $P(s>0)$ is the posterior sign
probability for the slope scar. ``fallback'' indicates whether a robust log$A$ overlap-domain
procedure was used to define the common summary domain across posterior draws; in this run suite,
all M2 runs required the fallback. The final row is an extended-redshift single-seed check with
$z_{\\max}=0.77$.}
\\label{tab:proxy_scars_multirun}
\\end{table}


\\paragraph{Interpretation (proxy stack).}
Across four clean M0 seeds, the mean deviation statistic remains small and consistent with zero
($m\\in[-0.079,-0.020]$), while the slope statistic remains preferentially negative
($s\\in[-0.623,-0.480]$) with $P(s>0)\\in[0.048,0.092]$ across seeds.
Across four M1 seeds, the slope magnitude is reduced relative to M0
($s\\in[-0.400,-0.182]$ with $P(s>0)\\in[0.037,0.195]$), supporting the interpretation that part of the
M0 tilt can be absorbed by residual closure freedom rather than by $\\mu(A)$ alone.
Across four M2 seeds, the slope remains negative with magnitude similar to M0
($s\\in[-0.531,-0.448]$), but all M2 runs require a robust overlap-domain construction for the scalar
summaries, reflecting curvature-induced variation in the horizon-area map across posterior draws.
Finally, an extended-redshift single-seed check to $z_{\\max}=0.77$ weakens the slope to
$s=-0.325\\pm0.275$, indicating that the slope scar is sensitive to the adopted redshift domain and
overlap region.

\\begin{remark}[M1 as a closure-degeneracy diagnostic]
Because M1 modifies the forward map through the product $\\mu(A(z))\\,[1+R(z)]$ (Eq.~\\eqref{eq:M1}),
background-only data cannot uniquely attribute a smooth redshift-dependent correction to the entropy
slope rather than to the closure residual. The observed reduction in $|s|$ under M1 is therefore
interpreted as a \\emph{systematic sensitivity} of the reconstructed tilt to mapping/closure
ambiguity, not as evidence that the M0 tilt is ``real'' microphysics. A key next step is to repeat
the full-likelihood ``info+'' suite under M1 (and M2) to test whether the negative-slope preference
persists once the closure freedom and growth/lensing anchors are combined.
\\end{remark}

\\begin{table}[h!]
\\centering
\\small
\\setlength{\\tabcolsep}{6pt}
\\begin{tabular}{@{}c r r r c@{}}
\\toprule
Map & $N_{\\mathrm{seed}}$ & mean($s$ mean) & sd across seeds & range($P(s>0)$) \\\\
\\midrule
M0 & 4 & $-0.522$ & 0.068 & $[0.048,0.092]$ \\\\
M1 & 4 & $-0.286$ & 0.095 & $[0.037,0.195]$ \\\\
M2 & 4 & $-0.492$ & 0.035 & $[0.055,0.117]$ \\\\
\\bottomrule
\\end{tabular}
\\caption{Seed-repeatability summary for the proxy configuration. These aggregates summarize
repeatability across random seeds for fixed data and priors; they are not independent measurements.}
\\label{tab:proxy_seed_repeat}
\\end{table}

\\begin{figure}[h!]
\\centering
\\includegraphics[width=0.95\\linewidth]{figure_A_logmu_band.png}
\\caption{Proxy-stack clean M0 base run (seed 123): posterior band for $\\log\\mu$ as a function of $\\log A$ over the overlap domain used for scalar summaries. The horizontal dashed line at $\\log\\mu=0$ indicates the Bekenstein--Hawking limit.}
\\label{fig:proxy_M0_seed123_logmu_band}
\\end{figure}

\\begin{figure}[h!]
\\centering
\\includegraphics[width=0.80\\linewidth]{figure_B_m_s_joint.png}
\\caption{Proxy-stack clean M0 base run (seed 123): joint posterior for the scar statistics $(m,s)$, with the Bekenstein--Hawking point $(0,0)$ marked.}
\\label{fig:proxy_M0_seed123_m_s_joint}
\\end{figure}


\\subsection{Info+ full-likelihood pilot (M0)}
\\label{sec:results_infoplus}
We completed a pilot ``info+'' multi-seed suite that augments the late-time geometry data with (i)
RSD $f\\sigma_8(z)$ points, (ii) the full Planck 2018 $C_\\ell^{\\phi\\phi}$ lensing bandpower
likelihood (via a CAMB backend), and (iii) a DR12 Shapefit full-shape monopole $P(k)$ likelihood.
The run output base is
\\texttt{outputs/finalization/info\\_plus\\_full\\_256\\_detached\\_20260129\\_0825UTC}.
The recorded Git SHA is \\texttt{cbc5bd4a646b31d263df02508a2c4abf62f80b91} with
\\texttt{dirty=True}; for publication-grade reproducibility this suite should be repeated on a clean
commit, but we report the numerical summaries as logged.

\\paragraph{Shared likelihood stack.}
All five seeds use mapping M0 over the automatically selected redshift domain $z\\in[0.02,0.62]$ and
the dataset stack in Table~\\ref{tab:datasets_infoplus}. The compressed lensing proxy likelihood is
disabled in this configuration to avoid double counting when the full $C_\\ell^{\\phi\\phi}$
bandpowers are included.

\\begin{table}[h!]
\\centering
\\footnotesize
\\setlength{\\tabcolsep}{4pt}
\\begin{tabularx}{\\textwidth}{@{}l c c X@{}}
\\toprule
Dataset component & points & redshift/scale & Source / notes \\\\
\\midrule
Pantheon+ (cosmology subset; stat+sys; $z_{\\mathrm{HD}}$) & 1322 (binned to 12) & $z\\in[0.02,0.62]$ & PantheonPlusSH0ES/DataRelease (commit \\texttt{c447f0f...}); full covariance with $M$-marginalization; binned for forward inference \\\\
Cosmic chronometers (BC03\\_all) & 9 & $z\\in[0.09,0.593]$ & baudren/montepython\\_public (tag 2.2); diagonal errors plus jitter $\\sigma_{\\mathrm{cc,jit}}$ \\\\
BAO-only (SDSS DR12 consensus BAO) & 6 & $z=\\{0.38,0.51,0.61\\}$ & CobayaSampler/bao\\_data (commit \\texttt{bb0c1c9...}); $(D_M/r_s,\\,H r_s)$ with full covariance \\\\
BAO-only (DESI 2024 ``ALL'') & 3 & $z=\\{0.295,0.51\\}$ & CobayaSampler/bao\\_data (commit \\texttt{bb0c1c9...}); $D_V/r_s$ and $(D_M/r_s,D_H/r_s)$ points in-range \\\\
RSD (DR16 LRG $f\\sigma_8$) & 2 & $z=\\{0.38,0.51\\}$ & Derived from the SDSS DR16 LRG FSBAO covariance source (CobayaSampler/bao\\_data) using the $f\\sigma_8$ sub-block \\\\
Planck 2018 lensing bandpowers & 9 & $\\ell\\in[8,400]$ & CobayaSampler/planck\\_supp\\_data\\_and\\_covmats (commit \\texttt{4c160c7...}); $C_\\ell^{\\phi\\phi}$ bandpowers with full covariance; CAMB backend \\\\
Full-shape $P(k)$ monopole (Shapefit) & 13 & $k\\in[0.026,0.145]\\,h\\,\\mathrm{Mpc}^{-1}$; $z_{\\mathrm{eff}}=0.51$ & Shapefit BOSS DR12 release (arXiv:2204.11868 data package); monopole-only block with covariance \\\\
\\bottomrule
\\end{tabularx}
\\caption{Datasets used in the info+ configuration. The compressed Planck lensing proxy is disabled here to avoid double counting when full $C_\\ell^{\\phi\\phi}$ bandpowers are included.}
\\label{tab:datasets_infoplus}
\\end{table}

\\paragraph{Shared sampler settings.}
Sampler: PTEMCEE with $n_T=4$ temperatures, $T_{\\max}=10$, and $N_w=64$ walkers.
Steps/burn/draws $=1500/500/800$ with $N_{\\mathrm{proc}}=52$ likelihood worker processes per seed.
Wall-clock time per seed is $\\sim 14.3$--$14.9$ hours in this run registry.

\\subsubsection{Cosmological summaries}
Table~\\ref{tab:infoplus_cosmo} reports marginal posterior summaries for core cosmological parameters.

\\begin{table}[h!]
\\centering
\\footnotesize
\\setlength{\\tabcolsep}{4pt}
\\begin{tabularx}{\\textwidth}{@{}r*{5}{>{\\centering\\arraybackslash}X}@{}}
\\toprule
Seed &
$H_0$ p50 [p16,p84] &
$\\Omega_{m0}$ p50 [p16,p84] &
$r_d$ [Mpc] p50 [p16,p84] &
$\\sigma_{8,0}$ p50 [p16,p84] &
$S_8$ p50 [p16,p84] \\\\
\\midrule
101 & 67.652 [66.307, 69.268] & 0.310 [0.286, 0.325] & 146.479 [141.574, 150.717] & 0.812 [0.796, 0.828] & 0.825 [0.789, 0.850] \\\\
202 & 68.085 [65.183, 87.650] & 0.255 [0.222, 0.279] & 150.617 [120.000, 160.938] & 0.795 [0.764, 1.000] & 0.765 [0.714, 0.874] \\\\
303 & 65.061 [62.922, 66.916] & 0.307 [0.293, 0.319] & 162.753 [154.488, 169.783] & 0.783 [0.754, 0.806] & 0.793 [0.748, 0.826] \\\\
404 & 67.664 [66.220, 69.438] & 0.295 [0.266, 0.313] & 148.424 [143.704, 152.497] & 0.806 [0.790, 0.824] & 0.799 [0.756, 0.831] \\\\
505 & 68.339 [66.623, 72.037] & 0.297 [0.258, 0.313] & 152.862 [146.497, 157.524] & 0.815 [0.798, 0.835] & 0.806 [0.765, 0.833] \\\\
\\bottomrule
\\end{tabularx}
\\caption{Info+ pilot (M0): posterior summaries by seed for core cosmological parameters. Values are
p50 medians with $[p16,p84]$ in brackets.}
\\label{tab:infoplus_cosmo}
\\end{table}

\\subsubsection{Scar statistics and diagnostics}
Table~\\ref{tab:infoplus_scars} reports the scalar scar summaries and sampler diagnostics.

\\begin{table}[h!]
\\centering
\\footnotesize
\\setlength{\\tabcolsep}{4pt}
\\begin{tabularx}{\\textwidth}{@{}r*{7}{>{\\centering\\arraybackslash}X}@{}}
\\toprule
Seed &
$m$ mean$\\pm$std &
$P(m>0)$ &
$s$ mean$\\pm$std &
$P(s>0)$ &
acc\\_mean &
ESS\\_min &
$\\tau_{\\max}$ \\\\
\\midrule
101 & $+0.1893\\pm 0.1182$ & 0.9313 & $-0.2105\\pm 0.4661$ & 0.3387 & 0.344 & 451.3 & 141.8 \\\\
202 & $-0.0644\\pm 0.2043$ & 0.3875 & $-0.3661\\pm 0.4339$ & 0.1875 & 0.327 & 466.4 & 137.2 \\\\
303 & $-0.2932\\pm 0.1384$ & 0.0200 & $-0.5074\\pm 0.4482$ & 0.1150 & 0.340 & 456.5 & 140.2 \\\\
404 & $+0.1359\\pm 0.1224$ & 0.8725 & $-0.1937\\pm 0.4756$ & 0.3375 & 0.349 & 449.4 & 142.4 \\\\
505 & $-0.1605\\pm 0.1108$ & 0.0612 & $-0.3938\\pm 0.4594$ & 0.2288 & 0.333 & 456.4 & 140.2 \\\\
\\bottomrule
\\end{tabularx}
\\caption{Info+ pilot (M0): scar summaries and sampler diagnostics by seed. $\\tau_{\\max}$ is the
largest estimated integrated autocorrelation time among a monitored set of 10 scalar projections.
In all seeds the post-burn chain length is shorter than $50\\,\\tau_{\\max}$, triggering the emcee IAT
warning and motivating the pilot-quality classification. Seed 202 used a global log$A$ overlap
fallback procedure for the scalar summaries; all other seeds used the strict overlap method.}
\\label{tab:infoplus_scars}
\\end{table}

\\paragraph{Summary interpretation.}
Across seeds, the weighted-mean scar $m$ is not stable in sign (two seeds positive, three negative)
and is consistent with zero at the across-seed level.
By contrast, the slope scar $s$ has a negative posterior mean in all five seeds. The mean of the
five per-seed slope means is $-0.334$ with an across-seed standard deviation $0.132$, and the
per-seed $P(s>0)$ values lie in the range $[0.115,0.339]$.
This is qualitatively consistent with the proxy-stack behavior (a negative tilt preference), but
the magnitude is reduced and the posterior for $s$ is broader.
The reduction in $|s|$ is expected when adding strong growth/lensing/full-shape anchors; it suggests that the
tilt is being \\emph{squeezed} by independent information rather than reflecting an unconstrained flexible mode.
However, given that M1 reduces $|s|$ in the proxy stack (Section~\\ref{sec:results_proxy}), an ``info+'' mapping
sensitivity test (M1/M2) is required before interpreting the remaining negative mean as anything beyond a
configuration-dependent consistency hint.


\\subsubsection{DES Y3 superstructures $\\times$ Planck CMB-lensing amplitudes (Tier 1; post hoc)}
\\label{sec:void_amp_tier1}
As a post-processing holdout check (no refitting), we map the five finished info+ M0 posteriors
(seeds 101/202/303/404/505) into a Tier~1, amplitude-only prediction for the DES Y3 void/supercluster
CMB-lensing amplitude parameter $A_\\kappa$ measured by Kov\\'acs et al.\\ \\cite{Kovacs2022Void}
(via DES Y3 superstructures $\\times$ Planck 2018 $\\kappa$).
The published quantity $A_\\kappa$ is an amplitude of a stacked lensing profile relative to a
simulation template and is \\emph{not} a direct measurement of $G_{\\mathrm{eff}}(z)$; consequently, the
mapping in Eq.~\\eqref{eq:void_amp_proxy} should be interpreted only as a fast directional
consistency probe under the minimal $\\alpha_M$-only embedding.

\\paragraph{Procedure.}
For each posterior draw $(j)$ and for each redshift bin $[z_{\\min},z_{\\max}]$ in
Ref.~\\cite{Kovacs2022Void}, we evaluate $\\mu^{(j)}(A(z))/\\mu^{(j)}(A(0))$ on a dense grid
($n_z=200$ points per bin) and form $\\mathcal{A}_{\\mathrm{void}}^{(j)}$ using
Eq.~\\eqref{eq:void_amp_proxy}.
This run uses a simple CMB-lensing-kernel-like redshift weight (\\texttt{weight=cmb\\_kappa})
computed from distances per draw and normalized on $[z_{\\min},z_{\\max}]$, with an effective source
comoving distance fixed to $14{,}000\\,\\mathrm{Mpc}$.
For bins extending beyond the stored posterior $z$ grid (notably $z_{\\max}=0.80$), we do not
extrapolate $\\mu$; instead we re-solve the background history $H(z)$ implied by the draw's $\\mu(A)$
out to $z_{\\max}$ and then evaluate $\\mu(z)$ on the extended grid.

\\paragraph{Aggregation across the five info+ runs.}
For each measurement bin, we summarize each seed run $r$ by its predicted mean
$\\hat{\\mathcal{A}}_r$ and predicted standard deviation $\\sigma_r$ across posterior draws, and report
a practical inverse-variance combination
\\begin{equation}
\\hat{\\mathcal{A}}_{\\mathrm{comb}}
\\coloneqq
\\frac{\\sum_{r=1}^{N_{\\mathrm{runs}}} \\hat{\\mathcal{A}}_r\\,\\sigma_r^{-2}}
{\\sum_{r=1}^{N_{\\mathrm{runs}}} \\sigma_r^{-2}},
\\qquad
\\sigma_{\\mathrm{comb}}^2
\\coloneqq
\\left(\\sum_{r=1}^{N_{\\mathrm{runs}}} \\sigma_r^{-2}\\right)^{-1},
\\label{eq:void_Acomb}
\\end{equation}
with $N_{\\mathrm{runs}}=5$ here. This is a convenient summary statistic, not a formally combined
posterior.

To provide an interpretable ``sigma-units'' summary (again, descriptive only), we also report
\\begin{equation}
Z_{\\mathrm{GR}}\\coloneqq \\frac{\\hat{\\mathcal{A}}_{\\mathrm{comb}}-1}{\\sigma_{\\mathrm{comb}}},
\\qquad
Z_{\\mathrm{obs}}\\coloneqq
\\frac{\\hat{\\mathcal{A}}_{\\mathrm{comb}}-\\mathcal{A}_{\\mathrm{obs}}}
{\\sqrt{\\sigma_{\\mathrm{comb}}^2+\\sigma_{\\mathcal{A}}^2}},
\\label{eq:void_Zscores}
\\end{equation}
where $\\mathcal{A}_{\\mathrm{obs}}\\pm\\sigma_{\\mathcal{A}}$ is the published amplitude for that bin.

\\begin{remark}[Reproducibility note (Tier 1 DES Y3 void proxy)]
This post-processing check is generated by \\path{scripts/run_void_amp_test.py} applied to the five
info+ M0 runs, using \\path{experiments/voids/data/void_lensing_amplitudes.json}, and written to
\\path{outputs/void_amp_test_infoplus_extz_20260130_032539UTC/} on git SHA
\\texttt{7f2cbc9e9ea1500c9819f04d91ee46bb93c48a03}. All 20 comparisons (5 runs $\\times$ 4 measurements)
completed successfully.
\\end{remark}

\\begin{table}[h!]
\\centering
\\footnotesize
\\setlength{\\tabcolsep}{4pt}
\\begin{tabularx}{\\textwidth}{@{}lcccccc@{}}
\\toprule
Bin & $[z_{\\min},z_{\\max}]$ &
$\\mathcal{A}_{\\mathrm{obs}}\\pm\\sigma_{\\mathcal{A}}$ &
$\\hat{\\mathcal{A}}_{\\mathrm{comb}}\\pm\\sigma_{\\mathrm{comb}}$ &
$Z_{\\mathrm{GR}}$ &
$Z_{\\mathrm{obs}}$ &
$\\langle\\Delta \\mathrm{LPD}\\rangle$ \\\\
\\midrule
Voids (low-$z$) & $[0.15,0.55]$ & $0.55\\pm0.23$ & $1.167\\pm0.098$ & $+1.70$ & $+2.47$ & $-0.37$ \\\\
Voids (high-$z$) & $[0.55,0.80]$ & $0.88\\pm0.13$ & $1.429\\pm0.266$ & $+1.61$ & $+1.86$ & $-1.13$ \\\\
Voids (all-$z$) & $[0.15,0.80]$ & $0.79\\pm0.12$ & $1.319\\pm0.193$ & $+1.65$ & $+2.33$ & $-0.35$ \\\\
Superclusters (low-$z$) & $[0.15,0.55]$ & $0.70\\pm0.15$ & $1.167\\pm0.098$ & $+1.70$ & $+2.61$ & $-0.22$ \\\\
\\bottomrule
\\end{tabularx}
\\caption{Tier~1 void/supercluster CMB-lensing amplitude proxy from the five info+ M0 posteriors,
compared to DES Y3 superstructures $\\times$ Planck 2018 lensing amplitudes from Kov\\'acs et al.\\
\\cite{Kovacs2022Void}. $\\hat{\\mathcal{A}}_{\\mathrm{comb}}\\pm\\sigma_{\\mathrm{comb}}$ is the inverse-variance
aggregated prediction defined in Eq.~\\eqref{eq:void_Acomb} (a practical summary, not a formally combined
posterior). $Z_{\\mathrm{GR}}$ and $Z_{\\mathrm{obs}}$ are defined in Eq.~\\eqref{eq:void_Zscores}.
$\\langle\\Delta \\mathrm{LPD}\\rangle$ is the mean log predictive density difference between the $\\mu(A)$-implied predictive distribution and the GR baseline $\\mathcal{A}=1$, averaged across the five runs.}
\\label{tab:void_amp_proxy}
\\end{table}

\\paragraph{Interpretation (Tier 1).}
Across bins, the Tier~1 mapping predicts $\\hat{\\mathcal{A}}_{\\mathrm{comb}}>1$, consistent in sign with
the negative-tilt behavior in $g(x)=\\log\\mu$ (which typically implies $\\mu(A(z))/\\mu(A(0))>1$ at higher
redshift/smaller mapped area under the $\\alpha_M$-only identification).
However, the DES Y3 superstructure amplitude measurements in Ref.~\\cite{Kovacs2022Void} are all below
unity, and the predictive score $\\langle\\Delta \\mathrm{LPD}\\rangle$ is negative for each bin, indicating
that (under this simplified amplitude proxy) the GR baseline $\\mathcal{A}=1$ tends to fit these
amplitudes better than the $\\mu(A)$-implied scaling.
Given that $A_\\kappa$ is an amplitude relative to a simulation template and depends on selection,
profile shape, and nonlinear structure, this should \\emph{not} be read as a decisive falsification of
$\\mu(A)$, but rather as an informative out-of-sample stress test that is sensitive to the assumed
perturbation-sector embedding.
A publication-grade void test would require a profile-based prediction matched to the measurement
definition and covariance, and an explicit treatment of screening and gravitational slip beyond the
Tier~1 amplitude-only mapping.


\\subsection{Post hoc proximity tests (proxy and info+; approximate)}
As a descriptive post-processing step, we compare the reconstructed $g(x)$ to simple parametric
families \\emph{after} inference (no parametric family is assumed in sampling). We use the weighted
function-space distance in Eq.~\\eqref{eq:D2} and an \\emph{approximate} pseudo-evidence difference
$\\Delta\\log Z$ defined in Eq.~\\eqref{eq:dlogZ}.

\\paragraph{Weighted function-space distance.}
For a parametric comparison family with log-deformation $g_{\\mathrm{model}}(x;\\vartheta)$, we define a
weighted squared distance
\\begin{equation}
D^2(\\vartheta)\\coloneqq
\\int_{x_{\\min}}^{x_{\\max}} w(x)\\,\\bigl[g(x)-g_{\\mathrm{model}}(x;\\vartheta)\\bigr]^2\\,\\dd x,
\\label{eq:D2}
\\end{equation}
and report the best-fit $\\vartheta$ (minimizing $D^2$) and the posterior mean of $D^2$ under the
reconstructed $g(x)$ samples.

\\paragraph{Approximate pseudo-evidence difference.}
Let $\\log Z_{\\mathrm{GP}}$ denote the log-evidence of a flexible nonparametric reference fit (GP), and
$\\log Z_{\\mathrm{model}}$ the corresponding quantity for a parametric family fit computed in the same
approximate scheme. We define
\\begin{equation}
\\Delta\\log Z_{\\mathrm{model}}\\coloneqq \\log Z_{\\mathrm{model}}-\\log Z_{\\mathrm{GP}}.
\\label{eq:dlogZ}
\\end{equation}
For the Monte-Carlo pseudo-evidence calculation, we use the explicit parameter priors implemented in the code: Tsallis $\\delta\\sim\\mathrm{Uniform}(0,2)$ and $\\log\\mu_0\\sim\\mathrm{Uniform}(-5,5)$; Barrow $\\Delta\\sim\\mathrm{Uniform}(-1,1)$ and $\\log\\mu_0\\sim\\mathrm{Uniform}(-5,5)$; Kaniadakis $\\log\\tilde\\beta\\sim\\mathrm{Uniform}(\\log 10^{-6},\\log 50)$ and $\\log\\mu_0\\sim\\mathrm{Uniform}(-5,5)$ (with reference area set to the median $A$ in the comparison domain).

\\begin{table}[h!]
\\centering
\\small
\\setlength{\\tabcolsep}{5pt}
\\begin{tabular}{@{}lcccc@{}}
\\toprule
Model family & best-fit parameter & $D^2$ (mean) & $\\Delta\\log Z$ vs GP & Note \\\\
\\midrule
BH ($\\mu=1$) & -- & $1.18\\times 10^{-3}$ & $-0.24$ & 0-parameter baseline \\\\
Tsallis & $\\delta=1.480$ & $2.64\\times 10^{-9}$ & $-4.28$ & power-law in $A$ \\\\
Barrow & $\\Delta=0.960$ & $2.64\\times 10^{-9}$ & $-4.30$ & power-law in $A$ \\\\
Kaniadakis & $\\tilde\\beta=0.752$ & $1.40\\times 10^{-6}$ & $-4.90$ & non-power-law family \\\\
\\bottomrule
\\end{tabular}
\\caption{Proxy-stack proximity metrics for the clean M0 base run (seed 123). The GP baseline is a
flexible nonparametric reference and generally has more effective degrees of freedom than the
low-dimensional parametric families. Consequently, $\\Delta\\log Z$ values include an Occam penalty and
should be interpreted only as an exploratory diagnostic (not as production-grade model selection).}
\\label{tab:prox}
\\end{table}

\\begin{table}[h!]
\\centering
\\small
\\setlength{\\tabcolsep}{4pt}
\\begin{tabular}{@{}lcc@{}}
\\toprule
Family & fitted parameter (mean$\\pm$sd across seeds) & $\\Delta\\log Z$ vs GP (mean across seeds) \\\\
\\midrule
Tsallis & $\\delta=1.33\\pm 0.13$ & $-3.07$ \\\\
Barrow & $\\Delta=0.67\\pm 0.26$ & $-2.55$ \\\\
Kaniadakis & $\\tilde\\beta=0.59\\pm 0.14$ & $-3.92$ \\\\
\\bottomrule
\\end{tabular}
\\caption{Info+ pilot proximity metrics aggregated across five seeds (Section~\\ref{sec:results_infoplus}). These pseudo-evidence differences are exploratory and are not treated as production-grade model selection, particularly given the pilot-quality convergence diagnostics in Table~\\ref{tab:infoplus_scars}.}
\\label{tab:prox_infoplus}
\\end{table}


% ============================================================
\\section{Conclusions}
We presented a calibrated, nonparametric pipeline to reconstruct an effective horizon-entropy slope
deformation $\\mu(A)$ from late-time cosmological data under explicit mapping variants (M0/M1/M2).
In a clean proxy-stack replacement run and associated robustness checks (Section~\\ref{sec:results_proxy}),
the weighted mean deviation statistic $m$ is consistent with zero, while the slope statistic $s$
shows a reproducible, configuration-dependent preference for negative values across four independent
M0 seeds. Multi-seed mapping-variant runs indicate that this slope preference is \\emph{mapping-sensitive}:
residual-closure freedom (M1) reduces the magnitude of the negative slope, while a curved-horizon area
map (M2) yields a slope similar to M0 but requires a robust overlap-domain procedure for scalar
summaries in all current M2 seeds. An extended-redshift single-seed check weakens the slope,
underscoring domain sensitivity.

In a pilot full-likelihood ``info+'' suite (Section~\\ref{sec:results_infoplus}) that includes RSD,
full Planck 2018 $C_\\ell^{\\phi\\phi}$ bandpowers, and a Shapefit full-shape $P(k)$ likelihood (all
under explicitly stated background-driven anchoring assumptions), the slope statistic
remains negative in all five seeds but with reduced magnitude and broader uncertainty than in the
proxy-stack configuration. The mean-deviation statistic remains mixed in sign and consistent with
zero. Integrated-autocorrelation warnings and occasional CAMB evaluation failures imply that these
info+ runs are pilot-quality until rerun with longer chains and validated with SBC/coverage tests.

As an initial out-of-sample stress test under the minimal $\\alpha_M$-only embedding (Section~\\ref{sec:minimal_embedding}), we mapped the info+ $\\mu(A)$ posteriors into a Tier~1 void/supercluster CMB-lensing amplitude proxy and compared to DES Y3 superstructures $\\times$ Planck 2018 lensing amplitudes (Section~\\ref{sec:void_amp_tier1}; Ref.~\\cite{Kovacs2022Void}). Under the amplitude-only mapping, the predicted proxy amplitudes lie above unity, while the published $A_\\kappa$ values are below unity and the log-predictive-density score favors the GR baseline $\\mathcal{A}=1$. This should be interpreted as an informative holdout check that is sensitive to the assumed perturbation-sector embedding and void-profile systematics, and motivates a profile-based extension before any strong physical claim.

In complementary dark-siren calibration work, we completed additional follow-up suites.
A fixed-power GR-truth injection grid (five injected log-$R$ scales, 256 replicates per scale)
shows the expected monotonic score response to injected propagation strength
($\\langle\\Delta\\mathrm{LPD}_{\\mathrm{tot}}\\rangle=\\{-0.495,-0.271,-0.019,+0.261,+0.562\\}$ for
scales $\\{0,0.5,1.0,1.5,2.0\\}$), with no replicate reaching
$\\Delta\\mathrm{LPD}_{\\mathrm{tot}}\\ge 3$.
A GR-systematics truth matrix (nine variants, 128 replicates each), a higher-stat six-variant pilot
(256 replicates each), and a targeted selection-sabotage matrix (four variants, 64 replicates each)
all remain subcritical relative to the real-data $\\Delta\\mathrm{LPD}_{\\mathrm{tot}}\\simeq +3.03$;
the largest observed GR-truth maximum in these suites is $+1.536$ (\\texttt{selection\\_weight\\_none}).
A subsequent high-fidelity adversarial+hierarchical checkpoint
(\\texttt{outputs/dark\\_siren\\_high\\_fidelity\\_20260208\\_025544UTC}) further tightened this bound:
an adversarial search over 12 plausibility-constrained systematics candidates (16 replicates each)
yields best-case mean $\\Delta\\mathrm{LPD}_{\\mathrm{tot}}=+0.132$ with maximum $+0.773$ and
$P(\\Delta\\mathrm{LPD}_{\\mathrm{tot}}\\ge 3)=0$, while hierarchical integration over five weighted
selection/incompleteness variants (24 aligned replicates) gives
$\\Delta\\mathrm{LPD}_{\\mathrm{tot,hier}}=+0.034\\pm0.174$ (maximum $+0.471$, again
$P(\\Delta\\mathrm{LPD}_{\\mathrm{tot}}\\ge 3)=0$), with a negative data term offset by a positive
selection term.
This high-fidelity run supersedes earlier fast-gate/smoke checkpoints and remains subcritical
relative to the real-data $\\Delta\\mathrm{LPD}_{\\mathrm{tot}}\\simeq +3.03$.
Detailed run-directory-level bookkeeping is reported in the reproducibility manifests and summary tables.
Pre-fix O4 sanity suites are excluded from inference; strict post-fix O4 runs are operationally valid but
model-sensitive and therefore calibration-limited rather than decision-grade.

These results are interpreted as phenomenological consistency information rather than as evidence for
a specific microphysical entropy model. In particular, the proxy-stack mapping sensitivity shows that allowing residual closure freedom (M1) can absorb part of the
negative tilt, so $s<0$ is not yet uniquely attributable to an entropy-slope deformation. At the background level, the reconstructed tilt is also degenerate with a smoothly evolving effective dark-energy sector; breaking this ``standard model pressure'' requires mapping-preference evidence comparisons and genuinely out-of-sample perturbation-sector
predictions (e.g.\\ standard sirens and local $\\dot G/G$ consistency). In this context, local $\\dot G/G$
constraints are treated as embedding-level gates (especially for unscreened completions), not as a
standalone falsification of cosmological propagation effects. The immediate priorities are:
(i) resolve the SBC under-coverage for scar summaries (Section~5) and repeat SBC for the configurations reported in Results,
(ii) rerun the full-likelihood info+ suite on a clean commit with longer chains and stabilized CAMB failure handling, and then extend it to M1/M2,
and (iii) expand out-of-sample cross-checks that do not enter the reconstruction likelihood and interpret them through explicit perturbation embeddings such as the minimal $\\alpha_M$-only option in Section~\\ref{sec:minimal_embedding}.

% ============================================================
\\section*{AI assistance disclosure}
The author used AI assistance throughout this project, including ChatGPT (OpenAI), for brainstorming, drafting and editing text, and iterating on analysis and software-development ideas.

% ============================================================
\\appendix
\\section{Clausius derivation of the forward $u(z)$ ODE}
\\label{app:clausius_derivation}
We derive Eq.~\\eqref{eq:forward_u} from a Cai--Kim/Clausius relation evaluated on the apparent horizon, following the conventions used in the code base.

For a flat FLRW background, the apparent-horizon radius is $R_A=c/H$ and the area is $A=4\\pi R_A^2$. The horizon temperature is taken to be
\\begin{equation}
T = \\frac{\\hbar c}{2\\pi k_B R_A}.
\\end{equation}
Using the Bekenstein--Hawking slope $(\\dd S/\\dd A)_{\\mathrm{BH}} = k_B c^3/(4G\\hbar)$ and the definition $\\mu(A)\\equiv (\\dd S/\\dd A)_{\\mathrm{BH}}/(\\dd S/\\dd A)$, we have
\\begin{equation}
\\dd S = \\frac{k_B c^3}{4G\\hbar}\\,\\frac{1}{\\mu(A)}\\,\\dd A.
\\end{equation}
Therefore,
\\begin{equation}
T\\,\\dd S
=
\\frac{\\hbar c}{2\\pi k_B R_A}\\;
\\frac{k_B c^3}{4G\\hbar}\\,\\frac{1}{\\mu(A)}\\,\\dd A
=
\\frac{c^4}{G}\\,\\frac{1}{\\mu(A)}\\,\\dd R_A,
\\label{eq:TdS_appendix}
\\end{equation}
where we used $\\dd A = 8\\pi R_A\\,\\dd R_A$.

Under the Cai--Kim/Clausius assumption, the heat flow across the apparent horizon in time $\\dd t$ is
\\begin{equation}
\\delta Q = A\\,(\\rho+p)\\,H\\,R_A\\,\\dd t.
\\label{eq:dQ_appendix}
\\end{equation}
Imposing $\\delta Q = T\\,\\dd S$ and combining Eqs.~\\eqref{eq:TdS_appendix}--\\eqref{eq:dQ_appendix} yields
\\begin{equation}
\\dot R_A = \\frac{4\\pi G}{c^4}\\,\\mu(A)\\,R_A^3\\,H\\,(\\rho+p).
\\end{equation}
For $R_A=c/H$, $\\dot R_A = -c\\,\\dot H/H^2$, which implies
\\begin{equation}
\\dot H = -\\frac{4\\pi G}{c^2}\\,\\mu(A)\\,(\\rho+p).
\\label{eq:Hdot_appendix}
\\end{equation}
In the late-time mapping we adopt a matter-dominance approximation $\\rho+p\\simeq\\rho_m(z)$ with
\\begin{equation}
\\rho_m(z)=\\rho_{m0}(1+z)^3,\\qquad \\rho_{m0}=\\frac{3H_0^2\\Omega_{m0}}{8\\pi G}.
\\end{equation}
Substituting into Eq.~\\eqref{eq:Hdot_appendix} gives
\\begin{equation}
\\dot H = -\\frac{3}{2}\\,H_0^2\\Omega_{m0}(1+z)^3\\,\\mu(A).
\\end{equation}
Finally, with $u\\equiv H^2$ and $\\dd z/\\dd t=-(1+z)H$, we obtain
\\begin{equation}
\\frac{\\dd u}{\\dd z}
=
\\frac{\\dd(H^2)/\\dd t}{\\dd z/\\dd t}
=
\\frac{2H\\dot H}{-(1+z)H}
=
3H_0^2\\Omega_{m0}(1+z)^2\\,\\mu\\!\\bigl(A(z)\\bigr),
\\end{equation}
which is Eq.~\\eqref{eq:forward_u}.

% ============================================================
\\begin{thebibliography}{99}

\\bibitem{SBC}
D.\\ Talts et al., ``Validating Bayesian inference algorithms with simulation-based calibration,''
\\emph{arXiv:1804.06788}.

\\bibitem{PTEMCEE}
M.\\ Vousden, W.\\ M.\\ Farr, and I.\\ Mandel, ``Dynamic temperature selection for parallel tempering in
Markov chain Monte Carlo simulations,''
\\emph{Mon.\\ Not.\\ R.\\ Astron.\\ Soc.} \\textbf{455} (2016) 1919--1937.

\\bibitem{CAMB}
A.\\ Lewis, A.\\ Challinor, and A.\\ Lasenby, ``Efficient computation of CMB anisotropies in closed FRW
models,'' \\emph{Astrophys.\\ J.} \\textbf{538} (2000) 473--476.

\\bibitem{Kovacs2022Void}
A.\\ Kov\\'acs et al., ``DES Y3 superstructures $\\times$ Planck 2018 CMB lensing: stacked void and
supercluster convergence profiles,'' \\emph{Mon.\\ Not.\\ R.\\ Astron.\\ Soc.} (2022).
arXiv:2203.11306, doi:10.1093/mnras/stac2011.

\\end{thebibliography}

\\end{document}
