\documentclass[aps,prd,twocolumn,superscriptaddress,nofootinbib]{revtex4-2}

\usepackage{amsmath,amssymb,bm}
\usepackage{graphicx}
\usepackage[colorlinks=true,linkcolor=blue,citecolor=blue,urlcolor=blue]{hyperref}
\usepackage{microtype}
\usepackage{lineno}

\graphicspath{{./figures/}{./}}

\newcommand{\dd}{\mathrm{d}}
\newcommand{\e}{\mathrm{e}}
\newcommand{\LPD}{\mathrm{LPD}}
\newcommand{\dL}{d_L}
\newcommand{\Mpl}{M_\ast}

\begin{document}
\title{A Calibrated Dark-Siren Tension with the General-Relativity Distance--Redshift Relation in GWTC-3}

\author{Aiden B. Smith}
\affiliation{Independent Researcher}

\begin{abstract}
Testing gravity at cosmological distances is now central to resolving late-time expansion tensions. We analyze 36 GWTC-3 dark sirens as a direct propagation test by comparing an internally fixed modified-propagation history against a General Relativity (GR) baseline. The data favor the modified-propagation model with a joint predictive-score difference $\Delta\LPD_{\rm tot}=+3.670$, corresponding to an evidence-ratio proxy $\exp(\Delta\LPD)\approx 39$ in this fixed scoring framework. Mechanism controls localize the effect to the distance--redshift channel: a sky-rotation null gives a comparable score distribution, and distance-only weighting retains most of the signal while sky-only weighting is subdominant. To quantify false alarms under GR, we run 512 GR-null catalog injections with the same event ensemble, incompleteness treatment, and empirically calibrated selection function. The null distribution is centered at $-0.839$ with width $0.240$, has maximum $+0.076$, and contains zero realizations with $\Delta\LPD\ge 3$. Thus the observed tension is far outside the calibrated GR-null ensemble generated by this pipeline. A nine-variant systematics matrix and a fixed-power response grid show that tested GR-consistent nuisances shift the score but do not reproduce the observed amplitude. The result is therefore a robust calibrated anomaly relative to the tested GR null, with interpretation bounded by remaining unmodeled selection and catalog systematics.
\end{abstract}

\maketitle
\linenumbers
\modulolinenumbers[5]

\section{Cosmological Context}
The luminosity-distance relation is one of the few direct ways to test gravity on cosmological baselines. In GR, gravitational-wave and electromagnetic luminosity distances are equal for the same background expansion history,
\begin{equation}
\dL^{\rm GW}(z)=\dL^{\rm EM}(z).
\end{equation}
In broad modified-gravity frameworks, an effective Planck-mass evolution produces
\begin{equation}
\dL^{\rm GW}(z)=R(z)\,\dL^{\rm EM}(z),
\qquad
R(z)=\frac{\Mpl(0)}{\Mpl(z)}.
\label{eq:Rdef}
\end{equation}
Dark sirens provide a population-level test of Eq.~\eqref{eq:Rdef} without requiring bright counterparts. This is timely because the late-time expansion sector remains under stress in precision cosmology.

\section{Data and Statistic}
We use 36 GWTC-3 dark sirens with public parameter-estimation samples and a host-incompleteness-marginalized galaxy-catalog likelihood. Selection is handled by an \emph{empirical selection function} trained from injections and applied consistently in data and null simulations.

For model $\mathcal{M}$ we define a joint predictive score over all events,
\begin{align}
\LPD(\mathcal{M}) &=
\log\!\left[
\frac{1}{N_s}\sum_{j=1}^{N_s}
\exp\!\left(
\sum_{i=1}^{N_{\rm ev}}\log p(d_i|\theta_j,\mathcal{M})
\right.\right. \nonumber \\
&\quad\left.\left.
-N_{\rm ev}\log\alpha(\theta_j,\mathcal{M})
\right)
\right].
\end{align}
and compare models with
\begin{equation}
\Delta\LPD_{\rm tot}=\LPD({\rm mod})-\LPD({\rm GR}).
\end{equation}
Here $\alpha$ is the selection normalization. Intuitively, larger $\LPD$ means better joint predictive fit to the observed event ensemble.

\section{Observed Tension in GWTC-3}
The observed score is
\begin{equation}
\Delta\LPD_{\rm tot}=+3.670,
\end{equation}
which corresponds to $\exp(\Delta\LPD)\approx 39$ in this fixed scoring setup.

Two controls identify the driving channel:
\begin{enumerate}
\item Sky-rotation null: random rotations of sky localization relative to the galaxy catalog yield a similar distribution ($\langle\Delta\LPD_{\rm rot}\rangle=+3.017$, sd $0.091$, with $P[\Delta\LPD_{\rm rot}\ge\Delta\LPD_{\rm real}]=0.45$).
\item Distance-vs-sky split: distance-only weighting retains most of the preference ($\Delta\LPD\simeq+2.995$), while sky-only weighting is smaller ($\Delta\LPD\simeq+0.969$).
\end{enumerate}
Thus the anomaly is primarily in the distance--redshift/selection sector, not unique host alignment geometry.

\section{Falsification of the GR Null Hypothesis}
\label{sec:null}
We compute the GR false-alarm behavior directly with 512 GR-null catalog injections using the same event ensemble, incompleteness model, and selection normalization used on real data. This yields
\begin{equation}
\langle\Delta\LPD_{\rm tot}\rangle=-0.839,\quad
\sigma=0.240,\quad
\max=+0.076,
\end{equation}
with zero injections at $\Delta\LPD\ge 3$.

Figure~\ref{fig:grnull} shows the key result: the observed score lies far outside the calibrated GR-null distribution generated by this pipeline.

\begin{figure*}[t]
\centering
\includegraphics[width=0.95\textwidth]{fig_delta_lpd_total_hist.png}
\caption{Calibrated GR-null ensemble (512 injections). Blue histogram: expected score distribution under GR for this analysis pipeline. Dashed orange: observed GWTC-3 value. The observed point lies far outside the GR-null range found in these injections (none with $\Delta\LPD\ge 3$).}
\label{fig:grnull}
\end{figure*}

\begin{figure*}[t]
\centering
\includegraphics[width=0.95\textwidth]{fig_delta_lpd_components_hist.png}
\caption{Score decomposition in the GR-null ensemble: data term and selection term. The net GR-null score remains negative, while the observed real-data score is positive and large.}
\label{fig:decomp}
\end{figure*}

\section{Systematics Stress Tests}
We test whether standard GR-consistent nuisance choices can generate the observed amplitude:
\begin{itemize}
\item Fixed-power response grid (5 injection scales, 256 replicates/scale): response is monotonic and directionally sensible, validating score sensitivity.
\item Nine-variant systematics matrix (128 replicates/variant): tested variants move the score but all maxima remain below $+1$ (largest $+0.678$), far below the observed $+3.670$.
\end{itemize}
Crucially, while selection effects are the primary suspect in dark-siren cosmology, these stress tests indicate that reproducing this specific amplitude requires selection-model errors substantially larger than those covered by standard calibration variations in the tested family.

\begin{figure*}[t]
\centering
\includegraphics[width=0.95\textwidth]{fig_fixed_power_grid.png}
\caption{Fixed-power response grid under the GR-null generator. Mean score increases with injected propagation power, confirming directional sensitivity of the statistic.}
\label{fig:power}
\end{figure*}

\begin{figure*}[t]
\centering
\includegraphics[width=0.95\textwidth]{fig_systematics_matrix.png}
\caption{Nine-variant GR-consistent systematics matrix. Tested nuisance variants shift $\Delta\LPD$, but none reproduce the observed high-amplitude anomaly.}
\label{fig:sysmat}
\end{figure*}

\begin{figure*}[t]
\centering
\fbox{\parbox{0.92\textwidth}{\centering Placeholder for Fig.~5 (to be replaced with final plot).}}
\caption{Fig 5: Reconstructed luminosity distance residuals ($d_L^{GW}/d_L^{EM}$) showing the preferred deviation from GR at $z > 0.5$.}
\label{fig:dlresid}
\end{figure*}

\section{Interpretation}
The central result is physically simple: \emph{within the tested null and nuisance families, the GWTC-3 dark-siren population is inconsistent with the GR propagation baseline used here}. Strikingly, this is not a marginal fluctuation around zero; it is an observed score far outside the calibrated GR-null ensemble generated with the same analysis machinery.

However, this should still be interpreted as a calibrated cosmological anomaly rather than a closed-form discovery claim. The dominant signal channel is distance--redshift/selection, so unmodeled catalog and selection effects outside the tested family remain a viable alternative explanation.

Why this matters for late-time cosmology is direct. If a propagation anomaly of this sign is real, analyses that assume GR propagation map GW amplitudes to distances with a systematic offset. That offset propagates into inferred expansion parameters and can bias standard-ruler or distance-ladder comparisons in the same direction as the observed Hubble-tension discrepancy. In that sense, the measurement is not only a gravity test; it is an explicit inference-bias mechanism candidate for part of the $H_0$ tension.

\section{Conclusion}
We report a large calibrated tension between GWTC-3 dark-siren data and the GR propagation baseline in this framework ($\Delta\LPD_{\rm tot}=+3.670$). A 512-run GR-null ensemble, matched to the same event and selection pipeline, does not reproduce this amplitude. Mechanism controls localize the effect to the distance--redshift/selection channel.

Crucially, while selection effects remain the leading concern in dark-siren cosmology, the tested stress matrix does not generate this amplitude within standard calibration ranges. This elevates the result from a software-level concern to a physically relevant anomaly in late-time inference.

The immediate implication is a concrete path for cosmology: either identify a larger, presently unmodeled selection/cross-calibration error that can bridge the gap, or treat propagation-sector modifications as an active ingredient in resolving part of the Hubble-tension inference mismatch.

\begin{acknowledgments}
This work used public GWTC-3 products and publicly available galaxy-catalog resources. Code and analysis artifacts are archived at Zenodo (DOI: 10.5281/zenodo.18604204).
\end{acknowledgments}

\begin{thebibliography}{99}

\bibitem{GWTC3}
R. Abbott \emph{et al.} (LIGO Scientific Collaboration, Virgo Collaboration, and KAGRA Collaboration),
\emph{Phys. Rev. X} \textbf{13}, 041039 (2023),
\href{https://doi.org/10.1103/PhysRevX.13.041039}{10.1103/PhysRevX.13.041039}.

\bibitem{Belgacem2018}
E. Belgacem, Y. Dirian, S. Foffa, and M. Maggiore,
\emph{Phys. Rev. D} \textbf{98}, 023510 (2018),
\href{https://doi.org/10.1103/PhysRevD.98.023510}{10.1103/PhysRevD.98.023510}.

\bibitem{Nishizawa2017}
A. Nishizawa,
\emph{Phys. Rev. D} \textbf{97}, 104037 (2018),
\href{https://doi.org/10.1103/PhysRevD.97.104037}{10.1103/PhysRevD.97.104037}.

\bibitem{GLADEplus}
G. D{\'a}lya \emph{et al.},
\emph{Mon. Not. R. Astron. Soc.} \textbf{514}, 1403 (2022),
\href{https://doi.org/10.1093/mnras/stac1443}{10.1093/mnras/stac1443}.

\bibitem{PlanckLensing2018}
Planck Collaboration,
\emph{Astron. Astrophys.} \textbf{641}, A8 (2020),
\href{https://doi.org/10.1051/0004-6361/201833886}{10.1051/0004-6361/201833886}.

\end{thebibliography}

\end{document}
