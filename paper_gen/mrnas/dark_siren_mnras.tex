\documentclass[fleqn,usenatbib]{mnras}

\usepackage[T1]{fontenc}
\usepackage{graphicx}
\usepackage{amsmath}
\usepackage{booktabs}

\graphicspath{{figures/}}

\newcommand{\LPD}{\mathrm{LPD}}
\newcommand{\dL}{d_L}

\title[GWTC-3 Dark-Siren Propagation Tension]{A robust calibrated dark-siren tension with General-Relativity propagation in GWTC-3}

\author[Aiden B. Smith]{
Aiden B. Smith$^{1}$\\
$^{1}$Data Analyst
}

\date{Submitted 2026 February 12}
\pubyear{2026}

\hypersetup{hypertexnames=false}

\begin{document}
\label{firstpage}
\pagerange{\pageref{firstpage}--\pageref{lastpage}}
\maketitle

\begin{abstract}
We report a calibrated propagation anomaly in 36 GWTC-3 dark sirens. Relative to an internal General Relativity (GR) propagation baseline, the modified-propagation hypothesis yields $\Delta\LPD_{\rm tot}=+3.670$, equivalent to a predictive-density ratio proxy $\exp(\Delta\LPD)\approx 39$ in this fixed scoring framework. The GR null is explicitly falsified within the calibrated pipeline: in 512 matched GR-consistent injections, none reaches the observed scale (0/512 with $\Delta\LPD\ge 3$; null maximum $+0.076$). Mechanism controls show that sky rotations preserve a high positive score distribution and therefore do not explain the anomaly, whereas distance--redshift controls retain most of the signal, localising the effect to the propagation/selection channel. A stress-test matrix of tested GR-consistent nuisance variants also fails to reproduce the amplitude (maximum $+0.678$). The result is therefore a robust calibrated tension, not a routine fluctuation in tested systematics. If this propagation-sector preference is physical, GR-locked distance inference defines a concrete bias pathway into late-time expansion fits, opening an immediate route to partial relief of the Hubble-tension budget.
\end{abstract}

\begin{keywords}
gravitational waves -- cosmology: observations -- cosmology: theory -- methods: statistical
\end{keywords}

\section{Introduction}
Late-time expansion remains under stress, most visibly in the persistent disagreement between local and early-Universe inferences of $H_0$. Most analyses target background expansion directly, but dark sirens probe a distinct sector: \emph{propagation}. In GR, the gravitational-wave luminosity distance equals the electromagnetic luminosity distance at fixed background cosmology,
\begin{equation}
\dL^{\rm GW}(z)=\dL^{\rm EM}(z),
\end{equation}
whereas modified-friction scenarios generically admit \citep{Belgacem2018,Nishizawa2018}
\begin{equation}
\dL^{\rm GW}(z)=R(z)\,\dL^{\rm EM}(z), \qquad R(z)=1 \ \text{in GR}.
\end{equation}
This Letter tests that propagation relation with GWTC-3 dark sirens using an internally calibrated posterior-predictive score and matched GR-null simulations.

\section{Data and Method}
We analyse 36 GWTC-3 dark sirens \citep{Abbott2023GWTC3} with a host-incompleteness-marginalised galaxy-catalogue likelihood using GLADE+ \citep{Dalya2022}. We restrict the analysis to GWTC-3 (O1--O3) and explicitly exclude O4a: O4a operated primarily as a two-detector network (LIGO Hanford and Livingston) due to the absence of Virgo and the limited duty cycle of KAGRA, which typically yields degenerate sky localisations as large annuli (``rings'') rather than compact triangulated regions. This inflates the galaxy-catalogue search volume and renders the dark-siren likelihood comparatively information-poor relative to the three-detector triangulation available in GWTC-3. Public PE samples are reweighted to remove the distance-prior imprint before scoring. The core statistic is the joint posterior-predictive log score, $\LPD(\mathcal{M})$, defined as
\begin{equation}
\begin{aligned}
\LPD(\mathcal{M}) \equiv {}&
\log\!\Bigl[
\frac{1}{N_s}\sum_{j=1}^{N_s}
\exp\!\Bigl(
\sum_{i=1}^{N_{\rm ev}}\log p(d_i \mid \theta_j,\mathcal{M})
\\
&\quad {}-N_{\rm ev}\log \alpha(\theta_j,\mathcal{M})
\Bigr)
\Bigl].
\end{aligned}
\label{eq:lpd_def}
\end{equation}
We then define the score difference between the modified-propagation hypothesis and the GR baseline as
\begin{equation}
\Delta\LPD_{\rm tot}\equiv \LPD({\rm prop})-\LPD({\rm GR}).
\label{eq:delta_lpd_def}
\end{equation}
The construction is internally calibrated: the same event ensemble, incompleteness treatment, and score definition are used for real data and null simulations. Crucially, the selection normalisation $\alpha$ is not a free phenomenological correction; it is empirically trained from injections (logistic selection model), which materially hardens the calibration.

\section{Results}
The updated O3 rerun gives
\begin{equation}
\Delta\LPD_{\rm tot}=+3.670
\end{equation}
with decomposition $\Delta\LPD_{\rm data}=+2.670$ and $\Delta\LPD_{\rm sel}=+1.000$. In this fixed score framework, this corresponds to $\exp(\Delta\LPD)\approx 39$.

Mechanism controls isolate the channel:
\begin{enumerate}
\item Sky-rotation null: $\langle\Delta\LPD_{\rm rot}\rangle=+3.017$ (s.d.\ 0.091) with $P(\Delta\LPD_{\rm rot}\ge\Delta\LPD_{\rm real})=0.45$, so the observed score is not a special sky-alignment outlier.
\item Distance--redshift versus sky split: distance-only retains most of the signal ($\Delta\LPD\simeq+2.995$), while sky-only is subdominant ($\Delta\LPD\simeq+0.969$).
\end{enumerate}
Since randomising sky coordinates preserves the positive score distribution, the tension is driven by the isotropic distance--redshift distribution (and its selection calibration), not by unique lines-of-sight or angular host clustering. Together, these controls indicate that the anomaly is driven by the distance--redshift/selection sector rather than by unique angular host matching.

\begin{figure*}
\centering
\includegraphics[width=\textwidth]{fig_dlresid_o3.png}
\caption{Residual reconstruction implying a redshift-dependent propagation offset relative to GR; the preferred trend provides a physical interpretation of the score excess if the effect is not due to catalogue/selection mismatch.}
\label{fig:residuals}
\end{figure*}

\begin{figure*}
\centering
\includegraphics[width=\textwidth]{fig_delta_lpd_total_hist.png}
\caption{The observed score lies far outside the calibrated GR-null distribution (512 injections), rejecting the GR null hypothesis within the injection-generator assumptions.}
\label{fig:null_hist}
\end{figure*}

\begin{figure*}
\centering
\includegraphics[width=\textwidth]{fig_delta_lpd_components_hist.png}
\caption{Decomposition of the calibrated GR-null ensemble into data and selection components, illustrating that the null remains centred negative even with a positive selection contribution.}
\label{fig:decomp}
\end{figure*}

\section{Discussion}
The stress-test programme materially constrains mundane explanations. In 512 GR-consistent injections, the null distribution is centred at $-0.839$ with width $0.240$, maximum $+0.076$, and 0/512 draws at $\Delta\LPD\ge 3$. A fixed-power injection grid gives the expected monotonic response, confirming directional sensitivity of the statistic. A nine-variant GR-consistent nuisance matrix shifts the score but never reproduces the observed amplitude (largest variant maximum $+0.678$).

These tests support a precise statement: within the tested nuisance family, the observed score is a robust calibrated tension with GR propagation. The leading caveat is also precise: unmodelled catalogue/selection errors outside this tested family can still contribute and remain the principal alternative explanation.

\begin{figure*}
\centering
\includegraphics[width=\textwidth]{fig_fixed_power_grid.png}
\caption{Injected-propagation power grid demonstrating a monotonic response of mean $\Delta\LPD_{\rm tot}$, validating that the statistic responds directionally to true propagation modifications.}
\label{fig:power_grid}
\end{figure*}

\begin{figure*}
\centering
\includegraphics[width=\textwidth]{fig_systematics_matrix.png}
\caption{GR-consistent systematics matrix showing that tested nuisance variants shift the score but do not reach the observed amplitude, motivating targeted expansion of the stress-test family.}
\label{fig:systematics}
\end{figure*}

\section{Conclusions}
GWTC-3 dark sirens now exhibit a calibrated propagation tension that is statistically large, internally consistent, and difficult to reproduce with tested GR-consistent nuisances. The null falsification is direct in this pipeline (0/512 at the observed scale), and mechanism controls identify the dominant channel as distance--redshift/selection rather than sky alignment.

If this preference reflects real propagation physics, assuming GR propagation in late-time inference becomes a built-in modelling error. In that interpretation, GR-locking acts as an invisible wedge that can bias expansion parameters and propagate directly into the Hubble-tension budget. The immediate scientific task is therefore binary and testable: either identify a larger unmodelled selection/catalogue effect that closes the gap, or promote propagation-sector freedom from optional extension to required baseline in precision late-time cosmology.

\section*{Data availability}
All numerical values quoted here are taken from the corresponding dark-siren production outputs in this repository.

\begin{thebibliography}{99}
\bibitem[Abbott et al.(2023)]{Abbott2023GWTC3}
Abbott, R., et al. (LIGO Scientific Collaboration, Virgo Collaboration, and KAGRA Collaboration) 2023, \emph{Phys. Rev. X}, 13, 041039, \doi{10.1103/PhysRevX.13.041039}

\bibitem[Belgacem et al.(2018)]{Belgacem2018}
Belgacem, E., Dirian, Y., Foffa, S., \& Maggiore, M. 2018, \emph{Phys. Rev. D}, 98, 023510, \doi{10.1103/PhysRevD.98.023510}

\bibitem[D\'alya et al.(2022)]{Dalya2022}
D\'alya, G., et al. 2022, \emph{Mon. Not. R. Astron. Soc.}, 514, 1403, \doi{10.1093/mnras/stac1443}

\bibitem[Nishizawa(2018)]{Nishizawa2018}
Nishizawa, A. 2018, \emph{Phys. Rev. D}, 97, 104037, \doi{10.1103/PhysRevD.97.104037}
\end{thebibliography}

\bsp
\label{lastpage}
\end{document}
